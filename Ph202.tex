%% LyX 2.0.6 created this file.  For more info, see http://www.lyx.org/.
%% Do not edit unless you really know what you are doing.
\documentclass[english]{article}
\usepackage[T1]{fontenc}
\usepackage[latin9]{inputenc}
\usepackage{geometry}
\geometry{verbose,tmargin=2.5cm,bmargin=3cm,lmargin=2.5cm,rmargin=2.5cm,headheight=2.5cm,headsep=2.5cm,footskip=2.5cm}
\usepackage{amsmath}
\usepackage{graphicx}
\usepackage{esint}
\PassOptionsToPackage{normalem}{ulem}
\usepackage{ulem}

\makeatletter

%%%%%%%%%%%%%%%%%%%%%%%%%%%%%% LyX specific LaTeX commands.
%% Because html converters don't know tabularnewline
\providecommand{\tabularnewline}{\\}
%% A simple dot to overcome graphicx limitations
\newcommand{\lyxdot}{.}


\makeatother

\usepackage{babel}
\begin{document}

\title{Ph20 Assignment \#2}


\author{Peter Lommen, TA: Jason Pollack}


\date{April 16th, 2014}

\maketitle

\section{Simpson's Approximation}


\subsection{Derivation of the extended Simpson's approximation}

We begin with Simpson's formula for a single interval $H=b-a$:
\[
I\simeq I_{simp}\equiv\dfrac{H}{2}\left(\dfrac{f(a)}{3}+\dfrac{4f(c)}{3}+\dfrac{f(b)}{3}\right).
\]
where $c=\dfrac{a+b}{2}$. We apply the same principles as in the
Trapezoid example: we divide $H$ by $N$ rather than 2 to get $h_{N}=\dfrac{b-a}{N}$.
Now we sum these smaller Simpson's approximations up, and see if anything
simplfies out:
\begin{eqnarray*}
\intop_{a}^{b}f(x)dx & = & \intop_{x_{0}}^{x_{2}}f(x)dx+\intop_{x_{2}}^{x_{4}}f(x)dx+\cdots+\intop_{x_{N-2}}^{x_{N}}f(x)dx\\
 & \simeq & h_{N}\left(\dfrac{f(x_{0})}{3}+\dfrac{4f(x_{1})}{3}+\dfrac{f(x_{2})}{3}\right)+h_{N}\left(\dfrac{f(x_{2})}{3}+\dfrac{4f(x_{3})}{3}+\dfrac{f(x_{4})}{3}\right)\\
+ & \cdots & +h_{N}\left(\dfrac{f(x_{N-2})}{3}+\dfrac{4f(x_{N-1})}{3}+\dfrac{f(x_{N})}{3}\right)
\end{eqnarray*}
Adding similar $f(x)$ terms together and simplifying, our final result
is thus:
\[
\intop_{a}^{b}f(x)dx\simeq h_{N}\left(\dfrac{1}{3}f(x_{0})+\dfrac{4}{3}f(x_{1})+\dfrac{2}{3}f(x_{2})+\dfrac{4}{3}f(x_{3})+\cdots+\dfrac{2}{3}f(x_{N-2})+\dfrac{4}{3}f(x_{N-1})+\dfrac{1}{3}f(x_{N})\right).
\]



\subsection{Local and Global Order of Simpson's Approximation}

We begin with the Taylor expansion of $f(x)$:
\[
f(x)=f(a)+f'(a)H+f''(a)\dfrac{H^{2}}{2}+f'''(a)\dfrac{H^{3}}{6}+f^{4}(\eta)\dfrac{H^{4}}{4!}
\]
As in the example with the Trapezoidal method, elementary integration
of this function yields
\[
I=f(a)H+f'(a)\dfrac{H^{2}}{2}+f''(a)\dfrac{H^{3}}{6}+f'''(a)\dfrac{H^{4}}{4!}+f^{4}(\eta)\dfrac{H^{5}}{5!}.
\]
We now use the Taylor expansion of $f(x)$ to rewrite Simpson's approximation:

\begin{eqnarray*}
I_{simp} & = & H\left(\dfrac{f(a)}{6}+\dfrac{4f(c)}{6}+\dfrac{f(b)}{6}\right)\\
 & = & H\left(\dfrac{f(a)}{6}+\dfrac{4}{6}\left(f(a)+f'(a)\dfrac{H}{2}+f''(a)\dfrac{H^{2}}{8}+f'''(a)\dfrac{H^{3}}{8\cdot6}+f^{4}(\eta)\dfrac{H^{4}}{16\cdot4!}\right)\right)\\
 &  & +\dfrac{1}{6}H\left(f(a)+f'(a)H+f''(a)\dfrac{H^{2}}{2}+f'''(a)\dfrac{H^{3}}{6}+f^{4}(\eta)\dfrac{H^{4}}{4!}\right)\\
 & = & f(a)H\left(\dfrac{1}{6}+\dfrac{4}{6}+\dfrac{1}{6}\right)+f'(a)H^{2}\left(\dfrac{1}{3}+\dfrac{1}{2}\right)+f''(a)H^{3}\left(\dfrac{1}{12}+\dfrac{1}{12}\right)\\
 &  & +f'''(a)H^{4}\left(\dfrac{1}{72}+\dfrac{1}{36}\right)+f^{4}(\eta)H^{5}\left(\dfrac{1}{6\cdot4\cdot24}+\dfrac{1}{6\cdot24}\right)\\
 & = & f(a)H+f'(a)\dfrac{H^{2}}{2}+f''(a)\dfrac{H^{3}}{6}+f'''(a)\dfrac{H^{4}}{24}+f^{4}(\eta)\dfrac{5H^{5}}{576}
\end{eqnarray*}
So if we subtract $I$ from $I_{simp}$, we get
\[
I_{simp}-I=f^{4}(\eta)H^{5}\left(\dfrac{5}{576}-\dfrac{1}{120}\right)=\dfrac{f^{4}(\eta)H^{5}}{2880}
\]
clearly showing that Simpson's approximation is of local order $O(H^{5})$.

Now for global order: as with the Trapezoidal method, the global error
is approximately the local error multiplied by $N$, if one replaces
$H$ with $h_{N}=\dfrac{b-a}{N}$. This works because over each subinterval
$[x_{i},x_{i+2}]$ the local error is again $\dfrac{f^{4}(\eta)h_{N}^{5}}{2880}$,
and there are $N$ subintervals, so the total error is close to 
\[
\dfrac{f^{4}(\eta)h_{N}^{5}}{2880}N=\dfrac{f^{4}(\eta)}{2880}\left(\dfrac{b-a}{N}\right)^{5}N=f^{4}(\eta)\dfrac{b-a}{2880}h_{N}^{4}.
\]
Therefore, the \uline{global error} of the extended Simpson's approximation
is of order $O(h_{N}^{5})$.


\section{Convergence}

\begin{center}
\includegraphics[scale=0.5]{Converge.png}
\par\end{center}

Above is a plot of the log of the absolute value of the error of the
$\mathtt{TrapInt}$ and $\mathtt{SimpInt}$ functions for varying
values of $N$. The Simpson's approximation converges much faster
than the Trapezoidal approximation, but around $N>1000$ becomes irregular;
it is here where the accuracy of the approximation method becomes
liable to roundoff errors; by default $\mathtt{Python}$ only keeps
17 significant figures, so it is here where the error is around $10^{-15}$
that roundoff errors become significant. The large spike in the error
for both the Trapezoidal and the Simpson's approximation is also likely
due to some bizarre roundoff error as well. These errors could possibly
(?) be fixed by changing some value in my code to a $\mathtt{float64}$
type.


\section{Relative Accuracy}

This wasn't a difficult function to write, but it's hard to analyze
quantitatively, because the Simpson's approximation converges very
quickly, and so you need to go for very small accuracy values to get
any significant value for $2^{k}N_{0}$. 


\section{Other methods of integration}

The errors of the Quad and Romberg method for approximating $\int_{0}^{1}e^{x}dx$
are compiled in the table below:

\begin{center}
\begin{tabular}{|c|c|c|c|}
\hline 
Method & Error Size & $N_{min}$ for Simpson's   & $N_{min}$ for Trapezoid\tabularnewline
\hline 
\hline 
Quad & $2.220446049250313\times10^{-16}$ & beyond point of rounding error & beyond point of rounding error\tabularnewline
\hline 
Romberg & $3.3084646133829665\times10^{-14}$ & 750 & $10^{9}$\tabularnewline
\hline 
\end{tabular}
\par\end{center}

To reach the equivalent level of accuracy specified by the quad method
is actually impossible for the given approximation method (without
the fixes mentioned above). To approximate the Romberg method requires
subintervals of around 750 for the Simpson's method, and around $10^{9}$
for the Trapezoidal method (once again, we see that Simpson's approximation
is $way$ better than the Trapezoidal method). 
\end{document}
